% Copyright 2004 by Till Tantau <tantau@users.sourceforge.net>.
%
% In principle, this file can be redistributed and/or modified under
% the terms of the GNU Public License, version 2.
%
% However, this file is supposed to be a template to be modified
% for your own needs. For this reason, if you use this file as a
% template and not specifically distribute it as part of a another
% package/program, I grant the extra permission to freely copy and
% modify this file as you see fit and even to delete this copyright
% notice. 

\documentclass{beamer}

% There are many different themes available for Beamer. A comprehensive
% list with examples is given here:
% http://deic.uab.es/~iblanes/beamer_gallery/index_by_theme.html
% You can uncomment the themes below if you would like to use a different
% one:
%\usetheme{AnnArbor}
%\usetheme{Antibes}
%\usetheme{Bergen}
%\usetheme{Berkeley}
%\usetheme{Berlin}
%\usetheme{Boadilla}
%\usetheme{boxes}
%\usetheme{CambridgeUS}
%\usetheme{Copenhagen}
%\usetheme{Darmstadt}
%\usetheme{default}
%\usetheme{Frankfurt}
%\usetheme{Goettingen}
%\usetheme{Hannover}
%\usetheme{Ilmenau}
%\usetheme{JuanLesPins}
%\usetheme{Luebeck}
\usetheme{Madrid}
%\usetheme{Malmoe}
%\usetheme{Marburg}
%\usetheme{Montpellier}
%\usetheme{PaloAlto}
%\usetheme{Pittsburgh}
%\usetheme{Rochester}
%\usetheme{Singapore}
%\usetheme{Szeged}
%\usetheme{Warsaw}

\title{Introduction to Git and GitHub}

% A subtitle is optional and this may be deleted
\subtitle{Source Control for COMP 167}

\author{Chris Cannon}
% - Give the names in the same order as the appear in the paper.
% - Use the \inst{?} command only if the authors have different
%   affiliation.

\institute[NCAT] % (optional, but mostly needed)
{
  Department of Computer Science\\
  North Carolina Agricultural and Technical State University
}
% - Use the \inst command only if there are several affiliations.
% - Keep it simple, no one is interested in your street address.

\date{Fall 2018}
% - Either use conference name or its abbreviation.
% - Not really informative to the audience, more for people (including
%   yourself) who are reading the slides online

\subject{Practical source control for computer science students}
% This is only inserted into the PDF information catalog. Can be left
% out. 

% If you have a file called "university-logo-filename.xxx", where xxx
% is a graphic format that can be processed by latex or pdflatex,
% resp., then you can add a logo as follows:

% \pgfdeclareimage[height=0.5cm]{university-logo}{university-logo-filename}
% \logo{\pgfuseimage{university-logo}}

% Delete this, if you do not want the table of contents to pop up at
% the beginning of each subsection:
\AtBeginSubsection[]
{
  \begin{frame}<beamer>{Outline}
    \tableofcontents[currentsection,currentsubsection]
  \end{frame}
}

% Let's get started
\begin{document}

\begin{frame}
  \titlepage
\end{frame}

\begin{frame}{Outline}
  \tableofcontents
  % You might wish to add the option [pausesections]
\end{frame}

% Section and subsections will appear in the presentation overview
% and table of contents.
\section{Git}

\subsection{What is the Purpose of Git?}

\begin{frame}{What is the Purpose of Git?}
  \begin{itemize}
  \item {
    Version Control
    \pause
  }
  \item {
    Prevents Data Loss
    \pause
  }
  \item {
	Allows Rollback  
	\pause
  }
  \item {
  	Iterative Development
  }
  \end{itemize}
\end{frame}

\subsection{How Git Works}

\begin{frame}{Blocks}
\begin{block}{Repository}
The repository is the authoritative final draft of work you have completed thus far.
\end{block}
\begin{block}{Staging Area}
The staging area is a rough draft, work that is quality, but incomplete.
\end{block}
\begin{block}{Working Directory}
The working directory is all files open related to the project, within the project folder.
\end{block}
\end{frame}

\subsection{Installing Git}

\begin{frame}{Installing Git}
  \begin{itemize}
  \item {
    Windows: www.git-scm.com
    \pause
  }
  \item {
    MacOS/Linux: terminal
    \pause
  }
  \item{
	git --version 
	\pause 
  }
  \item{
	sudo apt-get install git-all  
  }
  \end{itemize}
\end{frame}

\section{GitHub}

\subsection{Purpose of GitHub}

\begin{frame}{Purpose of GitHub}
\begin{itemize}
  \item {
    Remote Backup
    \pause
  }
  \item {
    Interactive Feedback
    \pause
  }
  \item{
	Collaboration 
	\pause 
  }
  \item{
	Public Portfolio 
  }
  \end{itemize}
  \end{frame}

\subsection{Creating a GitHub Account}

\begin{frame}{Creating a GitHub Account}
www.github.com
\end{frame}

\subsection{Working with GitHub}

% Placing a * after \section means it will not show in the
% outline or table of contents.
\section*{Questions}

\begin{frame}{Questions?}
\end{frame}



% All of the following is optional and typically not needed. 
\appendix
\section<presentation>*{\appendixname}
\subsection<presentation>*{For Further Reading}

\begin{frame}[allowframebreaks]
  \frametitle<presentation>{For Further Reading}
    
  \begin{thebibliography}{10}
    
  \beamertemplatebookbibitems
  % Start with overview books.

  \bibitem{ChaconStraub2018}
    Chacon and Straub
    \newblock {\em Pro Git}.
    \newblock APress, 2018.

  \end{thebibliography}
\end{frame}

\end{document}


